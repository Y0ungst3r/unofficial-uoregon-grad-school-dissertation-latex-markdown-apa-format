%--- Chapter 3 ----------------------------------------------------------------%
\chapter{Experimental Setup}
This section will describe the experimental setup used to collect the data necessary for the long lived particle searches presented in sections \ref{ch:llp_search_dt} and \ref{ch:llp_search_dedx_dt}. 
All the work done on these thesis was done with results from the ATLAS detector located at the Large Hadron Collider (LHC) at CERN.


\section{The Large Hadron Collider}
The Large Hadron Collider (LHC) is currently the world's largest and highest energy particle accelerator, with a total circumference of 27 kilometers.
The LHC is situated 100 meters underground, straddling the border between Switzerland and France near Geneva. 
The LHC accelerates protons to nearly the speed of light using radiofrequency cavities and superconduction magnets. 
The LHC collides the proton bunches at four interactions points evenly spaced around the accelerator rings. 
The energy of the proton-proton collisions at the interaction points are 13 TeV for run 2 and 13.6 TeV for run 3.
Protons are gathered from hydrogen gas 
\section{The ATLAS Detector}
