%--- Chapter 3 ----------------------------------------------------------------%
\chapter{Experimental Setup}
This section will describe the experimental setup used to collect the data necessary for the long lived particle searches presented in sections \ref{ch:llp_search_dt} and \ref{ch:llp_search_dedx_dt}. 
All the work done on these thesis was done with results from the ATLAS detector located at the Large Hadron Collider (LHC) at CERN.


\section{The Large Hadron Collider}
The Large Hadron Collider (LHC) is currently the world's largest and highest energy particle accelerator, with a total circumference of 27 kilometers.
The LHC is situated 100 meters underground, straddling the border between Switzerland and France near Geneva. 
The LHC accelerates protons to nearly the speed of light using radiofrequency cavities and superconduction magnets. 
The LHC collides the proton bunches at four interactions points evenly spaced around the accelerator rings.
At each interaction point sit one of the four main detectors, ATLAS, CMS, LHCb, and ALICE. 
The energy of the proton-proton collisions at the interaction points are 13 TeV for run 2 and 13.6 TeV for run 3.


The protons start as hydrogen gas and are accelerated via multiple different accelerators.
As hydrogen is a diatomic molecule the first step of accelerating protons is to actually add electrons to create $H^-$ ions. 
Then a strong electric field strips away an electron from each ion to create the protons aka $H^+$ ions.
The protons are then accelerated with RF cavities and focused with quadrupole magnets in what is called the LINAC4.
By the end of the LINAC4 the Protons reach an energy of 160 MeV.
Next step on the journey to collision is the Proton Synchrotron Booster (PSB) which further accelerates the protons to an energy of 2 GeV.
The Protons are then  \ref{bartosik2022performancelhcinjectorchain}

Finally the bunches are injected into the LHC where the counter-circulating beams collide.
\section{The ATLAS Detector}
The ATLAS detector is a general purpose particle detector at one of the four interaction points of the LHC. 
The ATLAS detector is 46 meters long, 25 meters in diameter, and weighs about 7000 tons and is made up of multiple subdetectors. 
Starting at the interaction point and moving outwards the subdetectors are the Inner Detector (ID), the Electromagnetic and Hadronic Calorimeters, and the Muon Spectrometer (MS).
Each of these subdetectors are designed to measure different properties and types of particles produced in the proton proton collisions.
The first subsystem is the Inner Detector (ID) which is the closest subdetector to the interaction point and is designed to measure the trajectories of charged particles.
The ID starts measuring particles at a radius of 33.5mm and extends to a radial distance of 
The next subsystem is the Electromagnetic calorimeter which causes electrons and photons to shower and deposit their energy. 
These showers are then measured to determine the energy of the original particles. 
Next is the Hadronic calorimeters which are designed to measure of the energy of hadrons that again shower in the system. 
The final subsytem is the Muon Spectrometer (MS) which is the outermost layer of the ATLAS detector and is designed to detect and measure the momentum of muons.
Muons are able to penetrate through the inner detector and calorimeters due to their high mass and weak interactions with matter, making the MS crucial for identifying and measuring muons produced in collisions.
The MS uses a combination of tracking chambers and magnetic fields to measure the curvature of muon trajectories, allowing for precise momentum determination.